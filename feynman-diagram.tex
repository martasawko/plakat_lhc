% Feynman diagram
% Requires PGF >= 2.0
\documentclass{article}

\usepackage[latin1]{inputenc}
\usepackage{tikz}
\usetikzlibrary{trees}
\usetikzlibrary{decorations.pathmorphing}
\usetikzlibrary{decorations.markings}
\usepackage{verbatim}

\begin{comment}
:Title: Feynman diagram
:Tags: Decorations, Trees, Physics & chemistry, PGF 2.0

I saw this `Feynman diagram`_ in `Edward Tufte's`_ book `Beautiful evidence` (you can also
find it in this thread_). It was relatively easy to recreate using trees and decorations.

**Update**: Rewritten using PGF 2.0 features. 


.. _Feynman diagram: http://scienceworld.wolfram.com/physics/FeynmanDiagram.html
.. _Edward Tufte's: http://www.edwardtufte.com/tufte/index
.. _thread: http://www.edwardtufte.com/bboard/q-and-a-fetch-msg?msg_id=0001TV&topic_id=1
.. _workaround: http://sourceforge.net/mailarchive/message.php?msg_id=15276319

\end{comment}


\begin{document}

% Define styles for the different kind of edges in a Feynman diagram
\tikzset{
    photon/.style={decorate, decoration={snake}, draw=red},
    electron/.style={draw=blue, postaction={decorate},
        decoration={markings,mark=at position .55 with {\arrow[draw=blue]{>}}}},
    gluon/.style={decorate, draw=magenta,
        decoration={coil,amplitude=4pt, segment length=5pt}} 
}

\begin{tikzpicture}[
        thick,
        % Set the overall layout of the tree
        level/.style={level distance=1.5cm},
        level 2/.style={sibling distance=2.6cm},
        level 3/.style={sibling distance=2cm}
    ]
    \coordinate
        child[grow=left]{
            child {
                node {$g$}
                % The 'edge from parent' is actually not needed because it is
                % implicitly added.
                edge from parent [gluon]
            }
            child {
                node {$g$}
                edge from parent [gluon]
            }
            edge from parent [gluon] node [above=3pt] {$g$}
        }
        % I have to insert a dummy child to get the tree to grow
        % correctly to the right.
        child[grow=right, level distance=0pt] {
        child  {
            child {
                child {
                    node {$\bar{d}$}
                    edge from parent [electron]
                }
                child {
                    node {$u$}
                    edge from parent [electron]
                }
                edge from parent [photon]
            }
            child {
                node {$b$}
                edge from parent [electron]
            }
            edge from parent [electron]
            node [below] {$t$}
        }
        child {
            child {
                node {$\bar{b}$}
                edge from parent [electron]
            }
            child {
                child {
                    node {$\bar{v}$}
                    edge from parent [electron]
                }
                child {
                    node {$e^{-}$}
                    edge from parent [electron]
                }
                edge from parent [photon]
            }
            edge from parent [electron]
            node [above] {$\bar{t}$}
        }
    };
\end{tikzpicture}


\end{document}