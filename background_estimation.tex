Using Monte Carlo we can well simulate the ratio of the background in the case of three quarks
with case of two quarks. 
We know that in the case of two $b$-quarks in the final state there is no chance of the occurrence of
the boson. 
Hence, even in real probe we will have only the background. 
The situation might be different in terms of case of our main interest --- that is in three quarks.
The real probe will consist of the background but also the potential signal.
We compare the ratio from the simulation with the ratio from the experiment and search for the 
extra energy.
