The Standard Model of particle physics is one of the most thoroughly verified theories in science.
The drawbacks of the Standard Model (SM) motivate search for beyond
Standard Model physics. The most popular theory explaining spontaneously-broken supersymmetry (SUSY) is called
the Minimal Supersymmetric Standard Model. It assumes that every particle has its superpartner and 
predicts the existence of the five Higgs bosons.
\textbf{ATLAS} experiment, which is run in Large Hadron Collider at CERN in Switzerland, studies the consequences of
Standard Model
and the nature of beyond Standard Model physics. The beams of protons are accelerated, collide and later are being 
examined in the detector.
We conducted our analysis using \textbf{ROOT}, which is a C++ library built by CERN.
The study focuses on eight types of probes in terms of signal masses which were hypothetical boson signals. 
We examine the final state of H/A boson decay --- one pair of $b\overline{b}$ and one $b$-quark.


