The analysis requires appling data reduction techniques during and after gathering data. The first
stage of data selection, run while collecting the data, is the \textbf{trigger} --- it decides about the event-accaptance 
and, in conssequences, reduces the background. The trigger has its three levels: L1, L2, EF. 
It selects by taking into the consideration things like energy, number
and direction of the jets.\\
Later, we begin offline reduction. It consists of specyfic \textbf{cuts} on some jets due to their 
transverse momentum, angle and $b$-taging. $b$-taging means ascribing events to particular cathegory:
\textit{bbb}(strongest demand about existance of third $b$-quark), \textit{bbloose}(less rigorous requirement concerning 
third $b$-quark, smaller likelihood that it will be found), \textit{bbanti}(the smallest probability [..]).